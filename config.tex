% Created 2022-03-16 Wed 22:21
% Intended LaTeX compiler: pdflatex
\documentclass[11pt]{article}
\usepackage[utf8]{inputenc}
\usepackage[T1]{fontenc}
\usepackage{graphicx}
\usepackage{longtable}
\usepackage{wrapfig}
\usepackage{rotating}
\usepackage[normalem]{ulem}
\usepackage{amsmath}
\usepackage{amssymb}
\usepackage{capt-of}
\usepackage{hyperref}
\author{Jiawei Chen}
\date{\today}
\title{}
\hypersetup{
 pdfauthor={Jiawei Chen},
 pdftitle={},
 pdfkeywords={},
 pdfsubject={},
 pdfcreator={Emacs 27.2 (Org mode 9.6)}, 
 pdflang={English}}
\begin{document}

\tableofcontents

;;; \$DOOMDIR/config.el -\textbf{- lexical-binding: t; -}-

;; Place your private configuration here! Remember, you do not need to run 'doom
;; sync' after modifying this file!

(use-package! cdlatex
  :when (featurep! +cdlatex)
  :hook (\LaTeX{}-mode . cdlatex-mode)
  :hook (org-mode . org-cdlatex-mode)
  :config
  ;; Use \(...\) instead of \$ \ldots{} \$.
  (setq cdlatex-use-dollar-to-ensure-math nil)
  ;; Disabling keys that have overlapping functionality with other parts of Doom.
  (map! :map cdlatex-mode-map
        ;; Smartparens takes care of inserting closing delimiters, and if you
        ;; don't use smartparens you probably don't want these either.
        ``\$'' nil
        ``('' nil
        ``\{'' nil
        ``['' nil
        ``|'' nil
        ``<'' nil
        ;; TAB is used for CDLaTeX's snippets and navigation. But we have
        ;; Yasnippet for that.
        (:when (featurep! :editor snippets)
          ``TAB'' nil)
        ;; AUCTeX takes care of auto-inserting \{\} on \uline{\^{} if you want, with
        ;; `\TeX{}-electric-sub-and-superscript'.
        ``\^{}'' nil
        ``}`` nil
        ;; AUCTeX already provides this with `\LaTeX{}-insert-item'.
        [(control return)] nil))

;; Some functionality uses this to identify you, e.g. GPG configuration, email
;; clients, file templates and snippets.
(setq user-full-name ``Jiawei Chen''
      user-mail-address ``jc5667@columbia.edu'')

;; Doom exposes five (optional) variables for controlling fonts in Doom. Here
;; are the three important ones:
;;
;; + `doom-font'
;; + `doom-variable-pitch-font'
;; + `doom-big-font' -- used for `doom-big-font-mode'; use this for
;;   presentations or streaming.
;;
;; They all accept either a font-spec, font string (``Input Mono-12''), or xlfd
;; font string. You generally only need these two:
;; (setq doom-font (font-spec :family ``monospace'' :size 12 :weight 'semi-light)
;;       doom-variable-pitch-font (font-spec :family ``sans'' :size 13))

;; There are two ways to load a theme. Both assume the theme is installed and
;; available. You can either set `doom-theme' or manually load a theme with the
;; `load-theme' function. This is the default:
;;(setq doom-theme 'doom-one)
(setq doom-theme nil)

;; If you use `org' and don't want your org files in the default location below,
;; change `org-directory'. It must be set before org loads!
(setq org-directory ``\textasciitilde{}/org/'')


;; This determines the style of line numbers in effect. If set to `nil', line
;; numbers are disabled. For relative line numbers, set this to `relative'.
(setq display-line-numbers-type 'relative)


(setq auto-save-default t)
;; Here are some additional functions/macros that could help you configure Doom:
;;
;; - `load!' for loading external *.el files relative to this one
;; - `use-package!' for configuring packages
;; - `after!' for running code after a package has loaded
;; - `add-load-path!' for adding directories to the `load-path', relative to
;;   this file. Emacs searches the `load-path' when you load packages with
;;   `require' or `use-package'.
;; - `map!' for binding new keys
;;
;; To get information about any of these functions/macros, move the cursor over
;; the highlighted symbol at press 'K' (non-evil users must press 'C-c c k').
;; This will open documentation for it, including demos of how they are used.
;;
;; You can also try 'gd' (or 'C-c c d') to jump to their definition and see how
;; they are implemented.


(with-eval-after-load 'ox-latex
(add-to-list 'org-latex-classes
             '(``org-plain-latex''
               ``$\backslash$\documentclass{article}
           [NO-DEFAULT-PACKAGES]
           [PACKAGES]
           [EXTRA]''
               (``$\backslash$\section{%s}'' . ``$\backslash$\section*{%s}'')
               (``$\backslash$\subsection{%s}'' . ``$\backslash$\subsection*{%s}'')
               (``$\backslash$\subsubsection{%s}'' . ``$\backslash$\subsubsection*{%s}'')
               (``$\backslash$\paragraph{%s}'' . ``$\backslash$\paragraph*{%s}'')
               (``$\backslash$\subparagraph{%s}'' . ``$\backslash$\subparagraph*{%s}''))))

(setq org-latex-listings 't)
(setq org-roam-directory ``\textasciitilde{}/Documents/roam'')
\end{document}
